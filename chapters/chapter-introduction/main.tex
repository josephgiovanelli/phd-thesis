\chapter{Introduction}
\label{chap:intro}

Artificial intelligence (AI) is currently the buzzword on everybody's lips.
Riding the wave of recent groundbreaking achievements, from self-driving cars \citep{} to intelligent chatbots \citep{}, AI is transforming industries and reshaping our daily lives.
Several interpretations and definitions have been provided over the years, yet the seminal perspective given by the Turing test \citep{turing1980computing} is still one worth mentioning.
\begin{definition}[Artificial Intelligence \citep{turing1980computing}]
A machine that shows intelligence indistinguishable from that of human beings is qualified to be labeled as artificial intelligence.
\end{definition}
This translates into understanding what falls under the umbrella of \textit{intelligence}, which is defined differently across the research areas as reasoning, planning, and learning.

\begin{definition}[Machine Learning \citep{samuel2000some}]
A machine with the ability to learn without being explicitly programmed.
\end{definition}
Building upon the notion of intelligence as learning, i.e. Machine Learning (ML), emerged as the pinnacle of AI due to its disruptive advancements.
At its core, ML aims at addressing problems for which the development of algorithms by humans is not feasible, because the algorithm itself is either not known or cost-prohibitive.
Examples include face recognition, fraud detection, sale forecasting, and object ranking.
The problems are solved instead by letting the algorithms (e.g., neural networks \cite{nn}) \textit{discover their own solutions}: they perform a training process atop a sample of historical data, borrowing techniques from disciplines such as numerical analysis, statistics, and information theory.
The training process consists of fitting internal parameters (e.g., weights and bias) and providing ML practitioners with a model, which is ready to ingest new data and tackle the problem at hand.

There exists a plethora of different algorithms solving the same problem with different strengths and weaknesses, confirming theoretical results proving that there is \textit{no silver bullet} \cite{kerschke2019automated}---no algorithm dominates all others in all respects.
Besides, algorithms often expose some hyperparameters controlling the learning behavior (e.g., learning rate).
To unleash the full potential of ML, practitioners have to carefully tune such hyperparameters but get easily overwhelmed by the showcased problem of combined algorithm selection and hyperparameter (CASH) optimization.

Automated machine learning (AutoML) demonstrates to play a crucial role in this landscape by tackling the CASH problem and, going beyond, by handling ever-larger search spaces in surprisingly small time budgets \cite{small_time_budgets}.
Remarkable milestones include bayesian optimization (BO) to explore promising configurations based on prior evaluations,
meta-learning (i.e., learning atop learning) to warm-start BO (i.e., to boost the convergence process) by suggesting configurations that worked well in previous similar cases, and multi-fidelity methods to partially evaluate time-consuming configurations.
Besides, off-the-shelf solutions \cite{auto_sklearn} are provided to tune entire ML pipelines, achieving -- in some cases -- higher performance than experts.

By lowering the barrier of access, AutoML emerged as promising for democratization of AI, i.e. making it accessible to both experts and non-experts alike.
Yet, when it comes to real-case scenarios, the journey of learning is riddled with challenges, ranging from the need for human intervention to mitigate harnessing, to the need for physical simulators.
% and heterogeneity of the data to constraints that may apply to the problem, from need for domain knowledge to .
While \Cref{chap:background} provides the necessary background, the remaining of the thesis investigates these challenges evolving into two parts.




\paragraph{Part I: Human-centered AutoML}

The original promise of AutoML was to automate certain ML tasks to a significant extent, thereby democratizing it and enabling non-experts to apply it in their respective domains.
However, despite their success, many current AutoML tools were not built around the user but rather around algorithmic ideas.
The stacking of complex mechanisms on top of each other unavoidably led to a less understanding of the process by the user -- even ML experts -- and allowing for very limited interaction.
Parts of the community have hence pushed towards a more human-centered AutoML process aimed at complementing, instead of replacing, human intelligence.
% This becomes even more crucial nowadays
% Motivated by ethical concerns and social bias issues arising at each step of the ML process, this becomes imperative in such a mitigating call nowadays.
Besides, motivated by ethical concerns and social bias issues arising at each step of the ML process, this approach becomes even imperative in such a mitigating call nowadays.
By placing the user back in the loop, it would be possible to revise and supervise the entire process, ensuring fairness, transparency, and ethical compliance.

In this part, we focus on the following contributions.
\begin{itemize}
    \item Providing the AutoML formalization to explicitly deal with thorough pipelines i.e., concatenations of pre-processing transformations and ML algorithms.
    \item Devising human-centered solutions for the main categories of ML i.e., supervised and unsupervised learning. The former is provided with a ground truth to drive the learning towards a user-specified objective function, the latter is by nature exploratory -- does not benefit from any ground truth -- and optimizing an objective is even more challenging.
    \item Addressing multi-objective ML, i.e. optimization of more than one loss or objective function, by interactively learning user preferences and drive the optimization towards it.
    \item Showcasing potential challenges, opportunities, and risks of novel (human-centered) AutoML interfaces with large language models (LLMs), i.e. AI models trained on large corpora of text to understand and produce human-like answers.
\end{itemize}

We organize this part as follows.
\Cref{automl-chap:formalization} provides the comprehensive AutoML formalization for ML pipelines.
\Cref{automl-chap:supervised} and \Cref{automl-chap:unsupervised} address supervised and unsupervised learning respectively.
\Cref{automl-chap:moo} delves into multi-objective ML and, lastly, \Cref{automl-chap:llm} discusses the novel human-centered AutoML interfaces with LLMs.
% In this part, we devise human-centered AutoML solutions for the main categories of ML.
% While in \Cref{automl-chap:formalization} we provide the AutoML formalization to explicitly deal with thorough pipelines i.e., concatenations of pre-processing transformations and ML algorithms, \Cref{automl-chap:supervised} and \Cref{automl-chap:unsupervised} address supervised and unsupervised learning respectively.
% The former is provided with a ground truth to drive the learning towards a user-specified objective function, the latter is by nature exploratory -- does not benefit from any ground truth -- and optimizing an objective is even more challenging.
% Then, \Cref{automl-chap:moo} delves into multi-objective ML, i.e. optimization of more than one loss or objective function.
% Lastly, \Cref{automl-chap:llm} showcases the potential challenges, opportunities, and risks of novel (human-centered) AutoML interfaces with large language models, i.e. AI models trained on large corpora of text to understand and produce human-like answers.

\paragraph{Part II: Physics-coupled AutoML}

As application areas go, there are domains in which ML models are not yet widespread.
Usually related to earth observations, the deployed applications tend to be particularly high-impact, trying to mitigate challenges such as climate change by delivering (more) eco-friendly systems.
Examples include weather forecasts and crop life-cycle management.
Domain experts leverage their knowledge to tune numerical simulators -- which encode well-known physical laws -- and deliver forecasts and analyses.
However, this process faces several challenges.
% several issues jeopardize this process.
The non-existence of universal well-defined practices translates to several trials and errors, with domain experts manually configuring the simulator parameters until an acceptable solution is found.
Yet, as variables involved in physics phenomena are subject to constant change, such numerical solution must undergo periodic (re-)calibration.
Besides, with the torrent of remotely sensed data available today, opportunities to integrate real-time observations in predictive models have emerged that traditional methods are not equipped to handle.
This is where advances in AI can push forward to enhance our understanding, provide support and, if necessary, steer the course of resolving global concerns to a more favorable future.
For instance, ML models exhibit singular performance in adapting themselves to handle scenarios different from the one they were trained on (e.g., fine-tuning of network parameters) and AutoML can significantly impact in automating several stages (e.g., tuning simulators and ML models).
% At the same time, relying on fully-automated data-driven methods for problems as complex and impactful as these is doomed to result in biased and unreliable.
% This underscores the ongoing importance of domain experts in this endeavor, highlighting their crucial role, with AI serving as a powerful tool.
% This is why domain experts, as always, remain crucial in this process, and why AI should be there as a powerful tool that supports them in fully taking advantage of the new opportunities we are presented with.

In this part, we focus on precision farming, which plays a pivotal role in addressing water wastage and improving crop efficiency. Specifically, we commit to the following contributions.
\begin{itemize}
    \item Devising an extensive and flexible architecture for a big-data smart-irrigation platform that allows to perform analytics by coupling (automated) machine learning with physically-based models.
    \item Controlling soil moisture in real-time by relying on a grid of sensors and building fine-grained 2D and 3D profiles that enable a comprehensive analysis of the field.
    \item Forecasting the soil moisture through a two-stage optimization technique that also enables the system to adapt to new in-situ conditions.
    \item Providing a smart-irrigation approach that estimates the daily water amount needed by the plant and -- based on a grid of sensors and weather forecasts -- schedules the irrigation.
\end{itemize}



We organize this part as follows. \Cref{precision-chap:formalization} provides the necessary formalization in precision farming and overviews the designed architecture of our big-data smart-irrigation platform.
\Cref{precision-chap:orchard} deepens on Orchard3D-Lab, a well-known crop and soil simulator enhanced with auto-tuning and data assimilation capabilities, i.e. ingestion of sensor data for more reliable estimations.
\Cref{precision-chap:pluto} and \Cref{precision-chap:forecasting} delve -- respectively -- into the modules of real-time soil moisture monitoring and forecasting.
Finally, \Cref{precision-chap:smart-irrigation} deepens the smart-irrigation approach.