\begin{abstract}

Abstract needed.

% The volume, variety, and high availability of data backing decision support systems have impacted on business intelligence, the discipline providing strategies to transform raw data into decision-making insights.
% Such transformation is usually abstracted in the ``knowledge pyramid,'' where data collected from the real world are processed into meaningful patterns.
% In this context, volume, variety, and data availability have opened for challenges in augmenting the knowledge pyramid.
% On the one hand, the volume and variety of \textit{unconventional} data (i.e., unstructured non-relational data generated by heterogeneous sources such as sensor networks) demand novel and type-specific data management, integration, and analysis techniques.
% On the other hand, the high availability of unconventional data is increasingly attracting data scientists with high competence in the business domain but low competence in computer science and data engineering; enabling effective participation requires the investigation of new paradigms to drive and ease knowledge extraction.
% The goal of this thesis is to augment the knowledge pyramid from two points of view, namely, by including unconventional data and by providing advanced analytics.
% As to unconventional data, we focus on mobility data and on the privacy issues related to them by providing (de-)anonymization models.
% As to analytics, we introduce a higher abstraction level than writing formal queries.
% Specifically, we design advanced techniques that allow data scientists to explore data either by expressing intentions or by interacting with smart assistants in hand-free scenarios.

\end{abstract}