% ******************************************************************************
% ****************************** Custom Margin *********************************

% Add `custommargin' in the document class options to use this section
% Set {innerside margin / outerside margin / topmargin / bottom margin}  and
% other page dimensions
\ifsetCustomMargin
  \RequirePackage[left=37mm,right=30mm,top=35mm,bottom=30mm]{geometry}
  \setFancyHdr % To apply fancy header after geometry package is loaded
\fi

% Add spaces between paragraphs
%\setlength{\parskip}{0.5em}
% Ragged bottom avoids extra whitespaces between paragraphs
\raggedbottom
% To remove the excess top spacing for enumeration, list and description
%\usepackage{enumitem}
%\setlist[enumerate,itemize,description]{topsep=0em}

% *****************************************************************************
% ******************* Fonts (like different typewriter fonts etc.)*************

% Add `customfont' in the document class option to use this section

\ifsetCustomFont
  % Set your custom font here and use `customfont' in options. Leave empty to
  % load computer modern font (default LaTeX font).
  %\RequirePackage{helvet}

  % For use with XeLaTeX
  %  \setmainfont[
  %    Path              = ./libertine/opentype/,
  %    Extension         = .otf,
  %    UprightFont = LinLibertine_R,
  %    BoldFont = LinLibertine_RZ, % Linux Libertine O Regular Semibold
  %    ItalicFont = LinLibertine_RI,
  %    BoldItalicFont = LinLibertine_RZI, % Linux Libertine O Regular Semibold Italic
  %  ]
  %  {libertine}
  %  % load font from system font
  %  \newfontfamily\libertinesystemfont{Linux Libertine O}
\fi

% *****************************************************************************
% **************************** Custom Packages ********************************

%\usepackage{algpseudocode}
% \RequirePackage[labelsep=space,tableposition=top]{caption}
% \renewcommand{\figurename}{Fig.} %to support older versions of captions.sty

% ******************************* Line Spacing *********************************

% Choose linespacing as appropriate. Default is one-half line spacing as per the
% University guidelines

% \doublespacing
% \onehalfspacing
% \singlespacing


% ************************ Formatting / Footnote *******************************
% Don't break enumeration (etc.) across pages in an ugly manner (default 10000)
%\clubpenalty=500
%\widowpenalty=500
%\usepackage[perpage]{footmisc} %Range of footnote options


% *****************************************************************************
% *************************** Bibliography  and References ********************

%\usepackage{cleveref} %Referencing without need to explicitly state fig /table

% Add `custombib' in the document class option to use this section
\ifuseCustomBib
   \RequirePackage[square, sort, numbers, authoryear]{natbib} % CustomBib

% If you would like to use biblatex for your reference management, as opposed to the default `natbibpackage` pass the option `custombib` in the document class. Comment out the previous line to make sure you don't load the natbib package. Uncomment the following lines and specify the location of references.bib file

%\RequirePackage[backend=biber, style=numeric-comp, citestyle=numeric, sorting=nty, natbib=true]{biblatex}
%\addbibresource{References/references} %Location of references.bib only for biblatex, Do not omit the .bib extension from the filename.

\fi

% ******************************************************************************
% ************************* User Defined Commands ******************************
% ******************************************************************************

% *********** To change the name of Table of Contents / LOF and LOT ************

%\renewcommand{\contentsname}{My Table of Contents}
%\renewcommand{\listfigurename}{My List of Figures}
%\renewcommand{\listtablename}{My List of Tables}


% ********************** TOC depth and numbering depth *************************

\setcounter{secnumdepth}{2}
\setcounter{tocdepth}{2}


% ******************************* Nomenclature *********************************

% To change the name of the Nomenclature section, uncomment the following line

%\renewcommand{\nomname}{Symbols}


% ********************************* Appendix ***********************************

% The default value of both \appendixtocname and \appendixpagename is `Appendices'. These names can all be changed via:

%\renewcommand{\appendixtocname}{List of appendices}
%\renewcommand{\appendixname}{Appndx}

% *********************** Configure Draft Mode **********************************

% Uncomment to disable figures in `draft'
%\setkeys{Gin}{draft=true}  % set draft to false to enable figures in `draft'

% These options are active only during the draft mode
% Default text is "Draft"
%\SetDraftText{DRAFT}

% Default Watermark location is top. Location (top/bottom)
%\SetDraftWMPosition{bottom}

% Draft Version - default is v1.0
%\SetDraftVersion{v1.1}

% Draft Text grayscale value (should be between 0-black and 1-white)
% Default value is 0.75
%\SetDraftGrayScale{0.8}


% *****************************************************************************
% ******************* Better enumeration my MB*************
% \usepackage{enumitem}
\usepackage{tabularx}
\usepackage{rotating}
\usepackage{pdfpages}
\usepackage{graphicx}
\usepackage{array}
\usepackage{cleveref}
\usepackage{booktabs}
\usepackage[utf8]{inputenc}
\usepackage{hyperref}
\usepackage{graphicx}
\usepackage{caption}
\usepackage{floatrow}
\usepackage{multirow}
\usepackage{xcolor}
\usepackage{colortbl}
\usepackage{algorithm}
\usepackage[noend]{algpseudocode}
\usepackage{amssymb}
\usepackage{amsmath}
\usepackage{pifont}
\usepackage{quoting}
\usepackage{cleveref}
\usepackage{epigraph}
\usepackage[flushleft]{threeparttable}
% \usepackage{subfig}
\usepackage{subcaption}
\usepackage[shortlabels]{enumitem}

\newtheorem{example}{Example}
\newtheorem{definition}{Definition}
\newtheorem{problem}{Problem}
\newtheorem{theorem}{Theorem}
\newtheorem{lemma}{Lemma}

% \newenvironment{proof}{{\bf Proof:}}{}

\newcommand{\cupdot}{\mathbin{\mathaccent\cdot\cup}}
\newcommand{\cmark}{\ding{51}}
\newcommand{\xmark}{\ding{55}}
\newcommand{\eop}{\hfill $\Box$}
\newcommand{\mf}[1]{\textcolor{red}{\textbf{#1} ??}}
\newcommand{\argmax}{\mathop{\mathrm{argmax}}}
\newcommand{\argmin}{\mathop{\mathrm{argmin}}}
\newcommand{\veryshortarrow}[1][3pt]{\mathrel{%
		\hbox{\rule[\dimexpr\fontdimen22\textfont2-.2pt\relax]{#1}{.4pt}}%
		\mkern-4mu\hbox{\usefont{U}{lasy}{m}{n}\symbol{41}}}}

\renewcommand{\sf}[1]{\textsf{\textup{#1}}}
\renewcommand{\tt}[1]{\texttt{\textup{#1}}}
\renewcommand{\pm}{\mathbin{\smash{\raisebox{0.35ex}{$\underset{\raisebox{0.5ex}{$\smash -$}}{\smash+}$}}}}

\algnewcommand\algorithmicforeach{\textbf{for each}}
\algdef{S}[FOR]{ForEach}[1]{\algorithmicforeach\ #1\ \algorithmicdo}

\newfloatcommand{capbtabbox}{table}[][\FBwidth]

\newif\ifproofread
\proofreadtrue % commentare questo per disattivare \mf \mg \eg \sr

\usepackage{listings}

\crefname{lstlisting}{listing}{listings}
\Crefname{lstlisting}{Listing}{Listings}

\lstset{%
    basicstyle=\small\ttfamily,
    frame=single,
    morecomment=[f][\color{Green}][0]{\#},
    tabsize = 4, %% set tab space width
    commentstyle = \color{green}, %% set comment color
    keywordstyle = \color{blue}, %% set keyword color
    stringstyle = \color{red}, %% set string color
    rulecolor = \color{black}, %% set frame color to avoid being affected by text color
    basicstyle = \small \ttfamily , %% set listing font and size
    breaklines = true, %% enable line breaking
    numberstyle = \tiny,
}

% \usepackage{booktabs}
% \usepackage{comment}
% \usepackage{amsmath}
% \usepackage{amsthm}
% \usepackage{amssymb}
% \usepackage{listings}
% \usepackage{hyperref}
% \usepackage{cleveref}
% \usepackage{caption}
% \usepackage{threeparttable,booktabs}
% \usepackage{algpseudocode}
% \usepackage{algorithm}
% \usepackage{balance}
% \usepackage[normalem]{ulem}
% \usepackage[dvipsnames]{xcolor}


% %%%%%%%%%%%%%%%%%%%%%%%%%%%%%%%%%%%%%%%%%%%%%%%%%%%%%%%%
% Uncomment to check unused bibitems from here
% \usepackage{refcheck}
% \makeatletter
% \newcommand{\refcheckize}[1]{%
%   \expandafter\let\csname @@\string#1\endcsname#1%
%   \expandafter\DeclareRobustCommand\csname relax\string#1\endcsname[1]{%
%     \csname @@\string#1\endcsname{##1}\wrtusdrf{##1}}%
%   \expandafter\let\expandafter#1\csname relax\string#1\endcsname
% }
% \makeatother
% \refcheckize{\cref}
% \refcheckize{\Cref}
% ... to here
% %%%%%%%%%%%%%%%%%%%%%%%%%%%%%%%%%%%%%%%%%%%%%%%%%%%%%%%%

% \usepackage{booktabs} % For formal tables
% \usepackage{fancyhdr,amsmath}             % AMS Math
% \usepackage[utf8]{inputenc}
% \usepackage{xr}
% \usepackage{fullpage}
% \usepackage{times}
% \usepackage[ruled,vlined,linesnumbered]{algorithm2e}
% \usepackage{mathrsfs}
% \usepackage{listings,xcolor}
% \usepackage{color}
% \usepackage{blindtext}
% \usepackage{makeidx}  % allows for indexgeneration
% \usepackage{graphicx}
% \usepackage{natbib}
% \usepackage{rotating}
% \usepackage[noend]{algpseudocode}
% \usepackage{paralist}
% \usepackage{tabularx}
% \usepackage{url}
% \usepackage{soul}
% \usepackage[disable]{todonotes}
% \usepackage{todonotes}
% \usepackage{enumitem}
% \usepackage{tikz,fullpage}
% \usepackage{hyperref}
% \usepackage{amssymb}
% \usepackage{tkz-berge}
% \usetikzlibrary{shapes.geometric}
% \usepackage{caption}
% \usepackage{subcaption}
% \usepackage{stmaryrd}
% \usepackage{courier}
% \usepackage{algorithm}
% \usepackage{eqparbox}
% \usepackage{float}
% \usepackage{subcaption}
% \usepackage{algorithmic}
